\section{Övningsfrågor}
\textbf{OBS!} Det finns en formelsamling i \vref{sec:formler}.

\begin{exercise}
    Vad är den främsta våglängden en stjärna med en yttemperatur på \SI{6000}{\kelvin} sänder ut?
\end{exercise}

\begin{exercise}
    Solen har en skenbar magnitud på $-27$. Sirius har en skenbar magnitud på $\num{-1.46}$. Hur mycket ljusstarkare är solen jämfört med Sirius?
\end{exercise}

\begin{exercise}
    En foton har frekvensen \SI{6e14}{\hertz}. Vilken energi har fotonen?
\end{exercise}

\begin{exercise}
    Solen har en luminositet på \SI{3.8e26}{\watt}. Hur mycket energi sänder solen ut på ett år?
\end{exercise}

\begin{exercise}
    Beräkna din egen Schwarzschildradie.
\end{exercise}

\begin{exercise}
    Utgå från att solen har luminositeten \qty{3.8e26}{W}, och yttemperaturen är \qty{5800}{K}. Antag att solen endast skickat ut fotoner med våglängden $\lambda_\text{max}$ från Wiens förskjutningslag. Hur många fotoner måste solen skicka per sekund för att uppnå den önskade luminositeten?
\end{exercise}

\begin{exercise}
    Bestäm hur mycket överskottsenergi det bildas om du fusionerar \ce{6p+} och \ce{6n} för att skapa en \ce{_6^12C} (vanligt kol). Ta hjälp av periodiska systemet i \vref{fig:periodic-table}. En atomisk massenhet (\qty{1}{u}) motsvarar \qty{\sim 1.661e-27}{kg}.
\end{exercise}

\begin{exercise}
    \textbf{Hardcore-uppgift:} Om vi antar att solen fusionerar endast via proton-proton kedjan i \vref{fig:proton-proton}, hur mycket helium-4 per sekund bildas ifall luminositeten av \qty{3.8e26}{W} endast kommer från överskottet från detta? Försumma biprodukternas energi, dvs. positronerna, neutrinerna och gamma-fotonerna. Ni kommer att behöva googla vissa massor (kolla \hyperlink{https://www.ptable.com/}{\textcolor{blue}{ptable.com}}).
\end{exercise}